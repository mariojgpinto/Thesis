%-------------------------------------------------------------------------------------
%-------------------------------------------------------------------------------------
%------------------------------------Introduction-------------------------------------
%-------------------------------------------------------------------------------------
%-------------------------------------------------------------------------------------
\chapter{Estado da Arte}
\label{chap:2-SoA}

Os primeiros sistemas de captura de informa��o em 3D remontam � d�cada de 1960 e estes usavam luzes, c�maras e projetores para realizar a tarefa. Era um processo moroso que exigia muito esfor�o e tempo para conseguir ter resultados satisfat�rios. Durante v�rios anos esta tecnologia n�o sofreu grandes desenvolvimentos e tal pode tamb�m ser justificado, por exemplo, pelas limita��es de largura de banda ou pela capacidade de armazenamento dispon�vel. No fim dos anos 1980 foram criados os primeiros scanners 3D a laser que usavam luz branca, lasers e sombras para capturar a superf�cie de objeto.
Desde ent�o a tecnologia tem evolu�do a passos largos e t�m surgido v�rios sistemas utilizando t�cnicas diferentes para o mesmo fim: fazer o scan 3D de informa��o. Apareceram v�rios sistemas com caracter�sticas distintas e como tal, com prop�sitos diferentes como a captura a longa ou curta dist�ncia, a aquisi��o de uma qualidade detalhada ou a prefer�ncia pela prototipagem r�pida. O aperfei�oamento e difus�o destes instrumentos serviu tamb�m como alavanca para algumas �reas como a Antropometria ou a preserva��o digital.
Atualmente j� existem v�rios dispositivos capazes de fazer aquisi��o 3d de forma f�cil e r�pida. Al�m dos sensores industriais orientados a capturas de grandes dimens�es, tamb�m j� existem sistemas que permitem fazer esse tipo de aquisi��es em casa. Produtos como o 