
\documentclass[12pt,a4paper]{book}

%------------------------------INTERNATIONALIZATION----------------------------
%\usepackage[utf8]{inputenc}
\usepackage[english]{babel} %\usepackage[english,portuges]{babel} % amazingly, the filename is really "portuges"! 
\usepackage[T1]{fontenc} % change from default OT1, which contains only the 128 ASCII characters.
\usepackage{textcomp}

%-----------------------------------TEXT, FONT---------------------------------
\usepackage{txfonts} % nicer fonts than TeX's default ones
\renewcommand{\sfdefault}{txss} %use txfont's sans serif font as default sans-serif font family
\renewcommand{\familydefault}{\sfdefault} %use sans-serif as default font family
\renewcommand{\footnotesize}{\fontsize{8}{10}\selectfont} % 8pt font size, 10pt leading (line height)
\renewcommand{\em}{\fontfamily{\rmdefault}\fontshape{it}\selectfont} % make emphasis use roman font
\renewcommand{\itshape}{\fontfamily{\rmdefault}\fontshape{it}\selectfont} % make italics use roman font

% Pretendido-----------------------------------SPACING AND MARGINS---------------------------------
\usepackage{setspace} % for easy configuration of line spacing
\usepackage{parskip} % separate paragraphs by adding vertical space between them
                     % instead of indenting the first line
\setlength{\parskip}{1em plus 0.5em minus 0.5em} % more spacing between paragraphs than the default set by parskip}
                                                 % the plus and minus values allow for some flexibility
                                                 % and help with preventing widows and orphand (see next line)
%\widowpenalty=300 % avoid "widows" (single lines at the end of a page)
%\clubpenalty=300 % avoid "orphans" (single lines at the beginning of a page)
\usepackage{a4wide} % more text, less margins
%\usepackage{fancyhdr} % to format headers and footers
\usepackage{longtable} % for tables that are too long to fit on a single page

%---------------------------CITATIONS AND BIBLIOGRAPHY-------------------------
\usepackage[square,colon]{natbib} % author-year bibliography format + many customization options
\renewcommand{\cite}{\citep} % for backwards compatibility: in case natbib is not used,
                             % most entries will be already using the standard \cite command
%\usepackage{footbib} % worth a try: places bibliography entries as footnotes; more user-friendly for long documets.

%---------------------------------FIGURES CONFIG--------------------------------
\usepackage{graphicx}
\usepackage{wrapfig}  % to allow images wrapped by text
\usepackage[font={small,it}]{caption} % to make all captions equally formatted
                                      % (figure, wrapfig, subfloat).
                                      % It MUST be loaded *before* package subfig
\usepackage{subfig}  % for producing single figures made up of several images, using \subfloat
\usepackage{sidecap} % for figures with captions on the side
%\graphicspath{{./Images/}}
\graphicspath{{./[00]Images/}} % use images from folder "images"
                           		 % without having to specify the path everytime an image is included
%\usepackage{hypcap} % To make links point to the top of floats (images and tables), 
                     % instead of pointing to the bottom, where the caption and label are.
                     % Remember to include \capstart on the floats
                     
%-----------------------------------PDF CONFIG----------------------------------
\usepackage[usenames,dvipsnames]{color}
\usepackage[pdfauthor={Mario Pinto, PG17664},%
            pdftitle={3D Concentric Data Acquisition using a Static Setup},%
            urlcolor=NavyBlue,% default is magenta
            citecolor=black,% default is green
            linkcolor=NavyBlue,% default is red
            pdftex,colorlinks,pagebackref]{hyperref} %,a4paper
\def\chapterautorefname{Chapter}       % Make hyperref's autoref names start with capital letters.
\def\sectionautorefname{Section}       % See url below for default values:
\def\subsectionautorefname{Subsection} % http://www.tug.org/applications/hyperref/manual.html#TBL-23
\def\tableautorefname{Table}           % Note: autoref feature didn't seem to work with tables

\usepackage{pdfpages} % to be able to include pdf documents with includepdf{}
%-----------------------------------END PACKAGES----------------------------------

\begin{document}

\onehalfspacing % 1.5 line height

%\includepdf{./pdf-include/thesis-cover.pdf}
\thispagestyle{empty} % prevent a page number to appear in the empty page below
\newpage % insert empty page (back of the cover)

\frontmatter %note: only available in the book class

% The \chapter command automatically sets \thispagestyle{plain},
% which places page numbers centered in the footer.
% Setting this pagestyle here makes the empty pages of the front matter
% follow the same format, for consistency (doing the reverse would be more work).
\pagestyle{plain}

%\includepdf{./pdf-include/thesis-folha-rosto.pdf}
%\includepdf{./pdf-include/declaracao.pdf}

\newcommand{\note}[1]{\textcolor{red}{#1}}
\newcommand{\kinect}{\emph{Kinect}}
\newcommand{\AR}{\emph{AR}}
\newcommand{\ar}{\AR}
\newcommand{\ARS}{\emph{AR-System}}
\newcommand{\ars}{\ARS}
\newcommand{\ARG}{\emph{AR-Game}}
\newcommand{\HMD}{\emph{HMD}}
\newcommand{\hmd}{\HMD}
\newcommand{\TODO}{\note{\emph{\indent Insert Text Here \indent}}}
\newcommand{\REF}{\note{\emph{\indent Reference Here \indent}}}
\newcommand{\SOA}{\emph{SoA}}
\newcommand{\soa}{\SOA}

%\include{./Text/[00]Head/[00][01]Thanks}
%\include{./Text/[00]Head/[00][02]Abstract}
%\include{./Text/[00]Head/[00][03]Resumo}

\setcounter{tocdepth}{2} % customize maximum depth level of table of contents
\tableofcontents

\setlength{\parskip}{3em} % silly hack to make the list of figures start with those of chapter 4 on a new page
\listoffigures
\setlength{\parskip}{1em plus 0.5em minus 0.5em} % end of silly hack

\listoftables

\mainmatter

%-------------------------------------------------------------------------------------
%-------------------------------------------------------------------------------------
%-------------------------------------Introdu��o--------------------------------------
%-------------------------------------------------------------------------------------
%-------------------------------------------------------------------------------------
\chapter{Introduction}
\label{chap:1-Introduction}

Era uma vez mal escrevido ups  
coiso.~\cite{Fishkin:2004}


\input{./[01]Introduction/[01][01]Motiva��o}
%%-------------------------------------------------------------------------------------
%-------------------------------------------------------------------------------------
%-------------------------------------Objetivos---------------------------------------
%-------------------------------------------------------------------------------------
%-------------------------------------------------------------------------------------
\section{Objetivos}
Objetivos.
%%-------------------------------------------------------------------------------------
%-------------------------------------------------------------------------------------
%-------------------------------Estrutura do Documento--------------------------------
%-------------------------------------------------------------------------------------
%-------------------------------------------------------------------------------------
\section{Estrutura do Documento}
Estrutura do Documento.

\input{./[02]SoA/[02][00]Introdu��o}
%-------------------------------------------------------------------------------------
%-------------------------------------------------------------------------------------
%---------------------------------M�todos de Captura----------------------------------
%-------------------------------------------------------------------------------------
%-------------------------------------------------------------------------------------
\section{M�todos de Captura}
M�todos de Captura.
%\input{./[02]SoA/[02][02]Sistemas_de_Aquisi��o_360}
%\input{./[02]SoA/[02][03]Aplica��es}
%\input{./[02]SoA/[02][04]Sum�rio}

%\input{./Text/[03]Methodology/[03][00]Introduction}
%\input{./Text/[04]Concept_Design/[04][00]Introduction}
%\input{./Text/[05]Development/[05][00]Introduction}
%\input{./Text/[06]Results_and_Discussion/[06][00]Introduction}
%\input{./Text/[07]Conclusion/[07][00]Conclusion}


\bibliographystyle{./[00]Bibtex/myplainnat} % Based on plainnat, which uses the full first names of the authors instead of initials;
                               % Differences from plainnat: 
                               % - Places last names before first names, to ease lookup
                               % - Uses quotes around titles to make them stand out better
\cleardoublepage % move to next page so that the link from the next command is added in the proper place
                 % Note: use \clearpage instead, if the book class is not used
                 % (or if the "oneside" argument is passed to the document class)
\phantomsection % create an anchor so that a link can be added in the ToC (see below)
\addtocounter{chapter}{1}
\addcontentsline{toc}{chapter}{\arabic{chapter}\hspace*{1em}\bibname} % add this section to the ToC

%\renewcommand{\bibname}{References} % to customize the title of the references section

%\nocite{*} % for debugging: displays all bibliography entries, even if they're not cited in the text.
\bibliography{./[00]Bibtex/references}

%\input{./Text/[08]Appendix/[08][00]Appendix}

\end{document}
